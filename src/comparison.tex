\section{Comparison of RSE-ops vs. DevOps}

The easiest way to start to map out the space of RSE-ops is to address a series of questions about people, goals, and practices, and make a direct comparison to DevOps. On a high level, RSE-ops has a stronger association with HPC, while DevOps has a stronger association with the cloud, but the lines are blurry. While early efforts of some of these clouds attempted to re-brand HPC \cite{google-hpc}, progress has been made to the point that the gap between cloud and HPC is narrowing, and HPC centers are able to take advantage of cloud technologies, and vice versa. There are still subtle differences, and ideally there could be convergence to empower researchers to use software across different platforms. 
For this reason, we think that making comparisons between the two can be helpful to understand what practices are well established for RSE-ops, and which require further development. Since there is a stronger association of HPC with RSE-ops, in the discussion below we will often be comparing HPC with cloud, however this does not say that there is always a strong dividing line between the two. We will proceed in the following sections to ask questions of each, speculate on best practices, and then summarize our findings in a table.

\subsection{What are the goals of each?}

Arguably, the goals of DevOps are to provide applications and services, typically web-based. The goals of RSE-ops are to sometimes provide services for scientific applications, but more-so to provide infrastructure and means to run, develop, and distribute, scientific software. RSE-ops, then, is for research software and services, while DevOps is typically for more widely available, persistent services and corresponding software. This does not mean, however, that RSEs are never involved with DevOps, nor that industry Software Engineers are never working on research software.

\subsection{Who is involved?}

You will typically find individuals practicing RSE-ops at academic institutions, national labs, and some private industry, or anywhere that high performance computing is the primary means of compute. While some companies might also use high performance computing, typically we likely find that larger companies maintain their own internal container orchestration system (e.g., Google uses Borg \cite{borg}, and smaller companies pay to use cloud services that offer a similar set of tooling. Likely this decision results from some cost-benefit analysis \cite{Prabhakaran2018-sn} that determines that one is more cost effective than the other. Whether we look at Google Cloud \cite{google-devops}, Microsoft Azure \cite{microsoft-devops} or Amazon Web Services \cite{aws}, all of these cloud environments have a primary focus on distributed, scaled, and "server-less" technologies. We might call this \href{https://en.wikipedia.org/wiki/Cloud\_native\_computing}{cloud computing}.

When we look closely at individuals involved, it tends to be the case that institutions with HPC have a combination of Linux Administrators, Support Staff, Research Software Engineers, and Researchers. The Research Software Engineers
in particular play an interesting role because they can sit on the administrative side (with Linux Administrators and Support Staff), on the user side (with Researchers) or somewhere in between. For this reason, they are essential staff for communication, or ensuring that the needs of the researchers are known by those that run the resources.
For tech companies, it's likely the case that a DevOps team or team of Support Reliability Engineers (SREs) is tasked with managing software and services for the company. The SREs are primarily concerned with how things should be done, and developing monitoring and other support tools, while a DevOps teams is primarily concerned with doing it \cite{google-sre}.
The line gets blurry with respect to titles, because a company can have some flexibility with respect to naming these roles. However, it's common to see titles like Software Engineer, DevOps Engineer, SRE, or even Cloud Architect. 
