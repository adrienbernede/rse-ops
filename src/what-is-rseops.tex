\subsection{What is Rse-ops?}

The role of Research Software Engineer (RSE) has been emerging in the last decade, and due to the hybrid nature of working environments across cloud, and HPC, DevOps practices are logically being adopted. So if DevOps is the intersection of  "Developer" and "Operations," then how does this concept map to this new space, where high performance computing, or more generally, Research Software Engineering is at the forefront?

Inspired by DevOps, we can define a similar term for the Research Software Engineering community to also inspire collaboration and champion best practices -- RSE-ops. Research Software Engineers (RSEs) \cite{rse-history} are those individuals that write code for scientific software, and more generally support researchers to use codes on high performance computing systems, cloud services, and lab computers.
Akin to traditional Software Engineers at major tech companies, they are responsible not just for software development, but also for deployment of analysis pipelines and general services.
It can be noted that we are not calling the new term RseDevOps (dropping "Dev"), and this is done intentionally as the term "Research Software Engineering" encompasses this "Development" portion. RSE-ops, then, appropriately refers to best practices for ensuring the same reliability, scale, collaboration, and software engineering for research codes. We may not always be running a scaled web service, but we might be running scaled jobs on a manager, profiling performance, or testing software in development.

Thus, RSE-ops is the intersection of Research Software Engineering and Operations, and generally refers to best practices for development and operations of scientific software.
Arguably, the RSE community has just as much to gain by building community and putting structure around these practices.
It's important to note that while high performance computing (HPC) has traditionally been a large part of scientific computation, researchers have extended their tools to also use cloud services and other non-HPC tools, so HPC is only considered a subset of Research Software Engineering and thus RSE-ops. Many modern applications are web-based and extend beyond HPC, and so it is important to consider this set as part of the larger scientific or research software engineering universe. However, the dual need to run or deploy application across environments presents greater challenges for the community.  

% We will want to make sure the document is clear on distinguishing between HPC and RSE - many RSEs use HPC, but not all, and HPC does not necessarily have research software engineers (they may be called developers or something else)
