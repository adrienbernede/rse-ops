\newgeometry{left=1cm,bottom=1cm,top=1cm}
\begin{minipage}{\linewidth}
\thispagestyle{empty}
\begin{center}
\centering
%\captionof{Table 1: Summary of Comparison between DevOps and RSE-ops}
%\newline \newline
\begin{tabular}{| p{15em}  p{15em}  p{15em}|} 
 \hline
 Description & DevOps & RSE-ops \\ [0.5ex] 
 \hline\hline
Goals & Commercial software and services & Research software development, use, and distribution \\
 \hline

People & Industry software engineers & Research software engineers \\
 \hline

Accessibility & Freedom to access from browsers, and anywhere with an internet connection & More secured access from possibly a limited set of internet connections  \\
 \hline

Maintenance & Request, use, and throw away when done & Constantly running resources that require maintenance and monitoring \\
 \hline

Staffing & No staffing required & Requires Linux administrators and user support specialists  \\
 \hline

Scientific Software & Software and services can be modular, and optimized for the application or service. & Requires complex software stacks with conflicting dependencies and variable architectures to co-exist on a resource. \\
 \hline

Scaling &  Scaling is typically automated. & Many options for scaling, and manual practices make it challenging for a cluster user or developer to know best practices.\\
  \hline

Software Distribution &  Complete freedom to use any software distribution or package manager. & Software is likely to come from external resources to be installed via a package manager or module system for the user. \\
\hline

Permissions & Complete freedom & Logically, only administrators can have elevated privileges to install software or otherwise interact with resources. \\
\hline

Accessibility & Browser and command line, even from mobile & Accessibility is primarily by way of the command line, with limited access to interactive notebooks. This is a huge potential area for development for rse-ops \\
 \hline

Testing & Automated testing and deployment alongside and integrated with cloud resources & Automated testing typically separate from the HPC resources \\

 \hline
Dependency Management & Easy to use bleeding edge software, and install only what you need when you need it & A hodge-podge of dependencies (versions and for different architectures) must co-exist on the resource \\

 \hline
Community Standards & Significant time and effort to establish standards for containers & Traditionally not as involved in the same efforts \\
 \hline

Continuous Integration & Well established practices and integration of version control with build, test, deploy & Limited interaction with traditional CI services, local deployment and custom runners is promising \\
 \hline

Continuous Deployment & Comes down to pushing containers to registries for production systems & No best practice established, but can interact with resources in some situations to deploy\\
 \hline

Monitoring & Monitoring is well integrated into services & Must "roll your own" monitoring, but DevOps services (e.g., Grafana, Prometheus) are used sometimes. \\
 \hline
Security & DevSecOps is leading the way to make security an automated part of the development lifecycle & Security is unlikely to be automated, and a greater challenge with many users sharing the same space. \\ [1ex] 
 \hline
\end{tabular}
\end{center}
\end{minipage}
\restoregeometry
\newgeometry{left=5cm,bottom=3cm,top=5cm,right=5cm}

% I'm not sure this is in scope to include here - it might just be too much.
% \section{HPC Technologies}

% There are a family of technologies that are a core part of high performance computing that are not present, or are represented in a different way, on the cloud.

% **TODO** add more detail here.
% we should talk about how this is a blurry line -- ParallelCluster is very similar to a regular
% HPC environment.
